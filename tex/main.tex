\documentclass[12pt,a4paper,parskip=full]{scrartcl}

%display boundaries
%\usepackage{showframe}

\usepackage{cite}
\usepackage[pdftex]{graphicx}
\graphicspath{{graphics/}}

\usepackage[cmex10]{amsmath}
\usepackage{amssymb}
\usepackage{listings}
\usepackage{tocloft}

\usepackage{url}

\usepackage{makeidx}
\usepackage[colorlinks,hyperindex,plainpages=false,
pdftitle={High-frequency characterization of a Software Defined Radio (SDR) platform},
pdfauthor={Gernot Vormayr},
pdfsubject={Bachelor thesis},
pdfkeywords={Hochfrequenz,Software Defined Radio,SDR,GNU Radio,USRP},
pdfpagelabels,
pagebackref,
bookmarksopen=false
]{hyperref}

\usepackage{cleveref}

\usepackage[binary-units]{siunitx}
\usepackage{pgfplots}

\pgfplotsset{compat=1.11}
\pgfplotsset{every axis plot/.append style={smooth}} 

\usepgfplotslibrary{units}
\pgfplotsset{unit code/.code 2 args={\si{#1#2}}}

\usepackage{tikz}
\usetikzlibrary{dsp,chains,fit}

\usepgfplotslibrary{external}
\tikzexternalize
\tikzsetexternalprefix{figures/}

\DeclareSIUnit \belm {Bm}
\DeclareSIUnit \belfs {BFS}
\DeclareSIUnit \samples {S}
\sisetup{per-mode = symbol}

\begin{document}
\begin{titlepage}
    \enlargethispage{1cm}
    \centering
    \begin{minipage}{0.49\textwidth}
        \includegraphics[height=1.5cm,keepaspectratio]{EMCE_Logo_CMYK_color}
    \end{minipage}
    \begin{minipage}{0.49\textwidth}
        \flushright
        \includegraphics[height=1.5cm,keepaspectratio]{TULogo_CMYK}
    \end{minipage}\\
    \vspace*{5cm}
    {\Huge \textbf{Bachelor thesis}}\\
    \vspace*{1cm}
    {\Large High-frequency characterization of a Software Defined Radio (SDR) platform}

    \vspace*{2cm}
    {\large Gernot ~\textsc{Vormayr} ~\\ 0425210 } ~\\ 

    \vfill
    {Supervisor} ~\\\vspace*{0.1cm}
    {Ass.Prof. Dipl.-Ing. Dr.techn. \large Holger ~\textsc{Arthaber}} ~\\
    {Univ.Ass. Dipl.-Ing. Dr.techn. \large Thomas ~\textsc{Faseth}}
    \vspace*{2cm}

    \rule{\textwidth}{0.4pt}
    \begin{minipage}[t]{0.70\textwidth}
        \begin{large}
            EMCE - Institute of Electrodynamics, Microwave and Circuit Engineering
        \end{large}\\
        Vienna University of Technology
    \end{minipage}
    \hspace{1pt}
    \begin{minipage}[t]{0.27\textwidth}
        Gusshausstrasse 25\\
        1040 Vienna\\
        www.emce.tuwien.ac.at
    \end{minipage}
    \vspace*{-3pt}
    \rule{\textwidth}{0.4pt}
    \clearpage
\end{titlepage}

\tableofcontents

\begin{abstract}
    The low cost Ettus Research USRP\footnote{Univeral Software Radio Platform} 
    combined with GNURadio enables implementing cheap and simple radio solutions.
    This allows rapid prototyping of new radio protocols, easier test setups for
    lab purposes and even reconstructing old analog communications hardware on a
    budget. Although Ettus Research provides schematics, source code, and documentation,
    there is no data available on the RF characteristics of the USRP devices.
\end{abstract}

\section{Introduction}
Software defined radio systems (SDR) can be operated on a very broad frequency range,
can be easily adapted to various needs and allow operations which would be
impossible or very expensive in analog hardware. A SDR consists of specialised
software, a general purpose computer, and an analog front end with fast
ADCs/DACs and optionally DSP and mixing capabilities. One of those front ends is
the Ettus Research USRP N2x0 product family. This family consists of the N200 and
the N210, which has a bigger FPGA, allowing additional custom DSP code directly
in the device. Both contain \SI{100}{\mega\samples\per\second} dual ADC,
\SI{400}{\mega\samples\per\second} dual DAC, Xilinx Spartan
3A-DSP, and Gigabit Ethernet connectivity to stream data. The analog part needs
to be provided by one of several possible daughter boards, which allow operation
from DC to 6 GHz\cite{ettus_n2x0}. This products are also available as rebranded
versions from NI.
\subsection{NI USRP-2922}
The NI USRP-2922 is a rebranded Ettus Research N210 with an SBX daughter board, which
allows frequency ranges from \SI{400}{\mega\hertz} to \SI{4.4}{\giga\hertz}. Both RX
and TX feature amplifiers and an attenuator to allow specifying an overall gain from
\SI{0}{\deci\bel} to \SI{31.5}{\deci\bel} in \SI{0.5}{\deci\bel} steps. Furthermore
both feature a clock synthesizer to generate the desired frequency from the \SI{100}{\mega\hertz}
base clock of the FPGA board, mixers and appropriate filtering. Digital to analog and analog to
digital conversion is handled by the base board at a fixed sampling rate of
\SI{100}{\mega\samples\per\second}. Samples can be transferred to and from the device via the
GBit network interface. This limits the maximum sampling rate for \SI{16}{\bit} to
\SI{25}{\mega\samples\per\second}. A detailed overview can be seen in \ref{fig:usrp}. %TODO wrong filter LP3 LP5; 20MHz?
\begin{figure}[ht]
    \centering
    \begin{minipage}{0.45\textwidth}
        \centering
        \includegraphics[scale=0.5]{overall}
    \end{minipage}
    \begin{minipage}{0.45\textwidth}
        \centering
        \includegraphics[scale=0.5]{sbx}
    \end{minipage}
    \caption{Overview of the hardware on the N210 circuit board and SBX ciruit board\cite{flo}.}
    \label{fig:usrp}
\end{figure}
Digital down conversion
\begin{figure}[htb]
    \centering
\resizebox{\textwidth}{!}{
\begin{tikzpicture}
    \matrix (m1) [row sep=5mm, column sep=5mm,ampersand replacement=\&]
    {
        \&
        \node[coordinate]                           (ma0) {};       \&
        \node[coordinate]                           (ma1) {};       \&
        \node[coordinate]                           (ma2) {};       \&
        \node[coordinate]                           (ma3) {};       \&
        \node[coordinate]                           (ma4) {};       \&
        \node[coordinate]                           (ma5) {};       \&
        \node                                       (ma6) {mag};    \\
        %------------------------------------------------------------
        \&
        \node[coordinate]                           (m00) {};       \&
        \node[coordinate]                           (m01) {};       \&
        \node[coordinate]                           (m02) {};       \&
        \node[dspsquare]                            (m03) {$\sum$}; \&
        \node[coordinate]                           (m04) {};       \&
        \node[coordinate]                           (m05) {};       \&
        \node[dspmixer]                             (m06) {};       \&
        \node[coordinate]                           (m07) {};       \&
        \node[coordinate]                           (m08) {};       \&
        \node                                       (m09) {phase};  \\
        %------------------------------------------------------------
        \&
        \node[dspnodeopen,dsp/label=above]          (m10) {adc\_a}; \&
        \node[coordinate]                           (m11) {};       \&
        \node[coordinate]                           (m12) {};       \&
        \node[dspadder,label=100:{-}]               (m13) {};      \&
        \node[dspnodefull]                          (m14) {};       \&
        \node[dspnodefull]                          (m15) {};       \&
        \node[dspadder]                             (m16) {};       \&
        \node[coordinate]                           (m17) {};       \&
        \node[coordinate]                           (m18) {};       \&
        \node[dspsquare]                            (m19) {$\sum$}; \&
        \node[coordinate]                           (m110){};       \&
        \node[dspsquare]                            (m111){CIC};    \&
        \node[dspsquare,label=above:small]          (m112){hb};     \&
        \node[dspsquare]                            (m113){hb};     \&
        \node[dspmixer]                             (m114){};       \&
        \node[coordinate]                           (m115){};       \\

        %------------------------------------------------------------
        \&
        \node[coordinate]                           (m20) {};       \&
        \node[coordinate]                           (m21) {};       \&
        \node[dspsquare]                            (m22) {sw};     \&
        \node[coordinate]                           (m23) {};       \&
        \node[coordinate]                           (m24) {};       \&
        \node[coordinate]                           (m25) {};       \&
        \node[coordinate]                           (m26) {};       \&
        \node[coordinate]                           (m27) {};       \&
        \node[dspsquare]                            (m28) {sw};     \&
        \node[dspsquare]                            (m29) {cordic}; \&
        \node[coordinate]                           (m210) {};   \&
        \node[coordinate]                           (m211) {};      \&
        \node[coordinate]                           (m212) {};      \&
        \node[coordinate]                           (m213) {};      \&
        \node                                       (m214) {scale}; \&
        \node[coordinate]                           (m215) {};      \&
        \&
        \node[dspnodeopen,dsp/label=above]          (m216) {sample}; \&
        \\
        %------------------------------------------------------------
        \&
        \node[dspnodeopen,dsp/label=below]          (m30) {adc\_b}; \&
        \node[coordinate]                           (m31) {};       \&
        \node[coordinate]                           (m32) {};       \&
        \node[dspadder,label=-100:{-}]              (m33) {};       \&
        \node[dspnodefull]                          (m34) {};       \&
        \node[coordinate]                           (m35) {};       \&
        \node[dspadder]                             (m36) {};       \&
        \node[coordinate]                           (m37) {};       \&
        \node[coordinate]                           (m38) {};       \&
        \node[coordinate]                           (m39) {};       \&
        \node[coordinate]                           (m310){};       \&
        \node[dspsquare]                            (m311){CIC};    \&
        \node[dspsquare,label=below:small]          (m312){hb};     \&
        \node[dspsquare]                            (m313){hb};     \&
        \node[dspmixer]                             (m314){};       \&
        \node[coordinate]                           (m315){};       \\
        %------------------------------------------------------------
        \&
        \node[coordinate]                           (m40) {};       \&
        \node[coordinate]                           (m41) {};       \&
        \node[coordinate]                           (m42) {};       \&
        \node[dspsquare]                            (m43) {$\sum$}; \&
        \node[coordinate]                           (m44) {};       \&
        \node[coordinate]                           (m45) {};       \&
        \node[dspmixer]                             (m46) {};       \\
        %------------------------------------------------------------
        \&
        \node[coordinate]                           (mb0) {};       \&
        \node[coordinate]                           (mb1) {};       \&
        \node[coordinate]                           (mb2) {};       \&
        \node[coordinate]                           (mb3) {};       \&
        \node[coordinate]                           (mb4) {};       \&
        \node[coordinate]                           (mb5) {};       \&
        \node                                       (mb6) {phase};  \\
    };
    \draw[dspline] (m09) -- (m19);
    \draw[dspconn] (m19) -- (m29);
    %connections start -> dc offset
    \foreach \off/\i in {1mm/1,-1mm/3}
    {
        \draw[dspflow] (m\i0) -- (m\i1);
        \draw[dspline] (m\i1) -- ([yshift=\off]m21) -- ([yshift=\off]m22.west);
        \draw[dspline] ([yshift=\off]m22.east) -- ([yshift=\off]m23);
        \draw[dspconn] ([yshift=\off]m23) -- (m\i3);
    }
    %top connection dc offset -> correction
    \begin{scope}[start chain]
        \chainin (m14);
        \chainin (m04) [join=by dspline];
        \chainin (m03) [join=by dspconn];
        \chainin (m13) [join=by dspconn];
        \chainin (m14) [join=by dspconn];
        \chainin (m15) [join=by dspconn];
        \chainin (m05) [join=by dspline];
        \chainin (m06) [join=by dspconn];
        \chainin (m16) [join=by dspconn];
    \end{scope}
    \draw[dspconn] (m15) -- (m16);
    \draw[dspline] (m16) -- (m17) -- ([yshift=1mm]m27);
    %bottom connection dc offset -> correction
    \begin{scope}[start chain]
        \chainin (m34);
        \chainin (m44) [join=by dspline];
        \chainin (m43) [join=by dspconn];
        \chainin (m33) [join=by dspconn];
        \chainin (m34) [join=by dspconn];
        \chainin (m36) [join=by dspconn];
        \chainin (m37) [join=by dspline];
    \end{scope}
    \draw[dspline] (m37) -- ([yshift=-1mm]m27);
    \draw[dspline] (m15) -- (m45);
    \draw[dspconn] (m45) -- (m46) -- (m36);
    \draw[dspline] (ma6) -- (m06);
    \draw[dspline] (mb6) -- (m46);
    %connection sw -> CIC
    \begin{scope}[dspconn]
        \draw ([yshift=1mm]m28.east) -- node[above] {i} ([yshift=1mm]m29.west);
        \draw ([yshift=-1mm]m28.east) -- node[below] {q} ([yshift=-1mm]m29.west);
    \end{scope}
    

    \foreach \off/\i in {1mm/1,-1mm/3}
    {
        \draw[dspconn] ([yshift=\off]m27) -- ([yshift=\off]m28.west);
        \draw[dspline] ([yshift=\off]m29.east) -- ([yshift=\off]m210) -- (m\i10);
        \begin{scope}[start chain]
            \chainin (m\i10);
            \foreach \j in {11,12,13,14}
            {
                \chainin (m\i\j)[join=by dspconn];
            }
        \end{scope}
        \begin{scope}[start chain]
            \chainin (m214);
            \chainin (m\i14)[join=by dspline];
            \chainin (m\i15)[join=by dspline];
            \chainin (m215)[join=by dspline];
        \end{scope}

    }
    \draw[dspflow] (m215) -- (m216);

    \node[draw,inner xsep=5pt,inner ysep=20pt,dashed,fit=(m03) (m13) (m14),label=above:{DC offset}] {};
    \node[draw,inner xsep=5pt,inner ysep=20pt,dashed,fit=(m43) (m33) (m34),label=below:{DC offset}] {};
    \node[draw,inner xsep=5pt,inner ysep=20pt,dashed,fit=(ma6) (mb6) (m15),label=below:{IQ balance}] {};
\end{tikzpicture}}
    \caption{DSP RX path}
    \label{fig:rxpath}
\end{figure}
\begin{itemize}
    \item ADC 14 Bit
    \item after DC 18 bit
    \item after corr 24 bit
    \item cordic 25 bit
    \item rest 24 bit
    \item 4 Stage CIC
    \item Round to nearest tith error diffusion
    \item DSP Core connection points!
\end{itemize}
\begin{figure}[htb]
    \centering
\resizebox{\textwidth}{!}{
\begin{tikzpicture}
    \matrix (m1) [row sep=5mm, column sep=5mm,ampersand replacement=\&]
    {
        \&
        \node[coordinate]                           (maa0) {};      \&
        \node[coordinate]                           (maa1) {};      \&
        \node[coordinate]                           (maa2) {};      \&
        \node[coordinate]                           (maa3) {};      \&
        \node[coordinate]                           (maa4) {};      \&
        \node[coordinate]                           (maa5) {};      \&
        \node[coordinate]                           (maa6) {};      \&
        \node[coordinate]                           (maa7) {};      \&
        \node[coordinate]                           (maa8) {};      \&
        \node[coordinate]                           (maa9) {};      \&
        \node                                       (maa10) {mag};  \\
        %------------------------------------------------------------
        \&
        \node[coordinate]                           (ma0) {};       \&
        \node[coordinate]                           (ma1) {};       \&
        \node[coordinate]                           (ma2) {};       \&
        \node[coordinate]                           (ma3) {};       \&
        \node[coordinate]                           (ma4) {};       \&
        \node[coordinate]                           (ma5) {};       \&
        \node                                       (ma6) {phase};  \&
        \node[coordinate]                           (ma7) {};       \&
        \node[coordinate]                           (ma8) {};       \&
        \node[coordinate]                           (ma9) {};       \&
        \node[dspmixer]                             (ma10) {};      \&
        \node                                       (ma11) {i\_dco};\\
        %------------------------------------------------------------
        \&
        \node[coordinate]                           (m00) {};       \&
        \node[coordinate]                           (m01) {};       \&
        \node[dspsquare]                            (m02) {hb};     \&
        \node[dspsquare,label=above:small]          (m03) {hb};     \&
        \node[dspsquare]                            (m04) {CIC};    \&
        \node[coordinate]                           (m05) {};       \&
        \node[dspsquare]                            (m06) {$\sum$}; \&
        \node[coordinate]                           (m07) {};       \&
        \node[dspmixer]                             (m08) {};       \&
        \node[dspnodefull]                          (m09) {};       \&
        \node[dspadder]                             (m010){};       \&
        \node[dspadder]                             (m011) {};      \&
        \node[coordinate]                           (m012) {};      \&
        \node[coordinate]                           (m013) {};      \&
        \node[coordinate]                           (m014) {};      \&
        \node[dspnodeopen,dsp/label=above]          (m015) {adc\_a};\&
        \\
        %------------------------------------------------------------
        \node[text width=4ex] {}; \&
        \node[dspnodeopen,dsp/label=left]           (m10) {sample}; \&
        \node[coordinate]                           (m11) {};       \&
        \node[coordinate]                           (m12) {};       \&
        \node[coordinate]                           (m13) {};       \&
        \node[coordinate]                           (m14) {};       \&
        \node[coordinate]                           (m15) {};       \&
        \node[dspsquare]                            (m16) {cordic}; \&
        \node[coordinate]                           (m17) {};       \&
        \node                                       (m18) {scale};  \&
        \node[coordinate]                           (m19) {};       \&
        \node[coordinate]                           (m110){};       \&
        \node[coordinate]                           (m111){};       \&
        \node[coordinate]                           (m112){};       \&
        \node[dspsquare]                            (m113){sw};     \&
        \node[coordinate]                           (m114){};       \\
        %------------------------------------------------------------
        \&
        \node[coordinate]                           (m20) {};       \&
        \node[coordinate]                           (m21) {};       \&
        \node[dspsquare]                            (m22) {hb};     \&
        \node[dspsquare,label=below:small]          (m23) {hb};     \&
        \node[dspsquare]                            (m24) {CIC};    \&
        \node[coordinate]                           (m25) {};       \&
        \node[coordinate]                           (m26) {};       \&
        \node[coordinate]                           (m27) {};       \&
        \node[dspmixer]                             (m28) {};       \&
        \node[coordinate]                           (m29) {};       \&
        \node[dspadder]                             (m210){};       \&
        \node[dspadder]                             (m211){};       \&
        \node[coordinate]                           (m212){};       \&
        \node[coordinate]                           (m213){};       \&
        \node[coordinate]                           (m214){};       \&
        \node[dspnodeopen,dsp/label=below]          (m215){adc\_b}; \&
        \\
        %------------------------------------------------------------
        \&
        \node[coordinate]                           (mb0) {};       \&
        \node[coordinate]                           (mb1) {};       \&
        \node[coordinate]                           (mb2) {};       \&
        \node[coordinate]                           (mb3) {};       \&
        \node[coordinate]                           (mb4) {};       \&
        \node[coordinate]                           (mb5) {};       \&
        \node[coordinate]                           (mb6) {};       \&
        \node[coordinate]                           (mb7) {};       \&
        \node[coordinate]                           (mb8) {};       \&
        \node[coordinate]                           (mb9) {};       \&
        \node[dspmixer]                             (mb10) {};      \&
        \node                                       (mb11) {q\_dco};\\
        %------------------------------------------------------------
        \&
        \node[coordinate]                           (mbb0) {};      \&
        \node[coordinate]                           (mbb1) {};      \&
        \node[coordinate]                           (mbb2) {};      \&
        \node[coordinate]                           (mbb3) {};      \&
        \node[coordinate]                           (mbb4) {};      \&
        \node[coordinate]                           (mbb5) {};      \&
        \node[coordinate]                           (mbb6) {};      \&
        \node[coordinate]                           (mbb7) {};      \&
        \node[coordinate]                           (mbb8) {};      \&
        \node[coordinate]                           (mbb9) {};      \&
        \node                                       (mbb10) {phase};\\
    };
    \draw[dspflow] (m10) -- (m11);
    \draw[dspline] (ma6) -- (m06);
    \draw[dspconn] (m06) -- (m16);
    \draw[dspline] (m01) -- node[above] {i} (m02);
    \draw[dspline] (m21) -- node[below] {q} (m22);
    \foreach \off/\i/\j in {1mm/0/a,-1mm/2/b}
    {
        \draw[dspline] (m11) -- (m\i1);
        \begin{scope}[start chain]
            \chainin (m\i2);
            \chainin (m\i3) [join=by dspconn];
            \chainin (m\i4) [join=by dspconn];
            \chainin (m\i5) [join=by dspline];
        \end{scope}
        \draw[dspline] (m\i5) -- ([yshift=\off]m15);
        \draw[dspconn] ([yshift=\off]m15) -- ([yshift=\off]m16.west);
        \draw[dspline] ([yshift=\off]m16.east) -- ([yshift=\off]m17);
        \draw[dspline] ([yshift=\off]m16.east) -- ([yshift=\off]m17) -- (m\i7);
        \draw[dspconn] (m\i7) -- (m\i8);
        \draw[dspline] (m18) -- (m\i8);
        \draw[dspline] (m09) -- (m\j9);
        \draw[dspconn] (m\j9) -- (m\j10);
        \begin{scope}[start chain]
            \chainin (m\j\j10);
            \chainin (m\j10)[join=by dspline];
            \chainin (m\i10)[join=by dspconn];
            \chainin (m\i11)[join=by dspconn];
            \chainin (m\i12)[join=by dspline];
        \end{scope}
        \draw[dspline] (m\j11) -- (m\i11);
        \draw[dspline] (m\i12) -- ([yshift=\off]m112);
        \draw[dspconn] ([yshift=\off]m112) -- ([yshift=\off]m113.west);
        \draw[dspline] ([yshift=\off]m113.east) -- ([yshift=\off]m114) -- (m\i14);
        \draw[dspflow] (m\i14) -- (m\i15);
    }
    \draw[dspline] (m08) -- (m09);
    \draw[dspconn] (m09) -- (m010);
    \draw[dspconn] (m28) -- (m210);


    \node[draw,inner xsep=5pt,inner ysep=20pt,dashed,fit=(m09) (maa10) (mbb10),label=below:{IQ Balance}] {};
    \node[draw,inner xsep=5pt,inner ysep=20pt,dashed,fit=(ma11) (mb11),label=below:{DC Offset}] {};
\end{tikzpicture}
}
    \caption{DSP TX path}
    \label{fig:txpath}
\end{figure}
\begin{itemize}
    \item adc\_a, adc\_b can be switched off
    \item DSP Core connection points!
\end{itemize}
\subsection{GNU Radio}
\begin{itemize}
    \item c++/python/grc
    \item volk init
    \item rx.grc tx.grc
    \item modified python versions (DC offset!)
\end{itemize}
\section{Measurement Setup}
\subsection{GNU Radio}
See rx/tx grc
\subsection{Matlab}
\begin{itemize}
    \item describe rx/tx
    \item describe measurement classes
    \item why not measurement toolbox?
\end{itemize}
\subsection{Meters}
\begin{itemize}
    \item R\&S SMBV 100a
    \item R\&S SMIQ06B \& Agilent E85810 GPIB
    \item R\&S ZVL
    \item Agilent N1914A (Power head!)
\end{itemize}
\section{Measurement Results}
\subsection{RF Frequency Response}
\subsubsection{RX}
\begin{itemize}
    \item Setup (smbv -> cable -> usrp)
    \item 1MHz offset (dc!)
\end{itemize}
\begin{figure}[htb]
    \centering
\begin{tikzpicture}
    \begin{axis}[
        legend pos = outer north east,
        legend style={font=\footnotesize},
        xlabel={gain},
        ylabel={power},
        x unit={\deci\bel},
        y unit={\deci\belm},
        minor tick num = 1
        ]
        \addplot[mark = +] table {data/rf/rx/8/dbm_400};
        \addplot[mark = x] table {data/rf/rx/8/dbm_2400};
        \addplot[mark = asterisk] table {data/rf/rx/8/dbm_4400};
        \legend{\SI{400}{MHz},\SI{2.4}{GHz},\SI{4.4}{GHz}}
    \end{axis}
\end{tikzpicture}
    \caption{Input power needed for full scale}
    \label{fig:inputfscrf}
\end{figure}
\begin{figure}[htb]
    \centering
\begin{tikzpicture}
    \begin{axis}[
        legend pos = outer north east,
        legend style={font=\footnotesize},
        ylabel={gain},
        xlabel={frequency},
        y unit={\deci\bel},
        x unit={\hertz},
        change x base,
        x SI prefix=giga,
        width=12cm,
        height=5cm,
        scale only axis,
        minor tick num = 1
        ]
        \addplot[mark = +] table {data/rf/rx/8/f_0};
        \addplot[mark = x] table {data/rf/rx/8/f_15};
        \addplot[mark = asterisk] table {data/rf/rx/8/f_30};
        \legend{\SI{0}{dBm},\SI{15}{dBm},\SI{30}{dBm}}
    \end{axis}
\end{tikzpicture}
    \caption{Gain needed for fullscale}
    \label{fig:gainfscrf}
\end{figure}
\begin{itemize}
    \item describe the drop from 400 -> 4.4Ghz (caused by amps)
    \item 31.5dB max
\end{itemize}
\subsubsection{TX}
\begin{itemize}
    \item usrp -> 16dB att (10dB + 6 dB) -> pwr meter
    \item measured cable
\end{itemize}
\begin{figure}[htb]
    \centering
\begin{tikzpicture}
    \begin{axis}[
        legend pos = outer north east,
        legend style={font=\footnotesize},
        ylabel={power},
        xlabel={frequency},
        y unit={\deci\belm},
        x unit={\hertz},
        change x base,
        x SI prefix=giga,
        width=12cm,
        height=5cm,
        scale only axis,
        minor tick num = 1,
        ymin = -25
        ]
        \addplot[mark = +] table {data/rf/tx/0.8/0.0};
        \addplot[mark = x] table {data/rf/tx/0.8/10.0};
        \addplot[mark = asterisk] table {data/rf/tx/0.8/20.0};
        \addplot[mark = triangle] table {data/rf/tx/0.8/30.0};
        \addplot[mark = 10-pointed star] table {data/rf/tx/0.8/31.5};
        \legend{\SI{0}{dB},\SI{10}{dB},\SI{20}{dB},\SI{30}{dB},\SI{31.5}{dB}}
    \end{axis}
\end{tikzpicture}
    \caption{Output power, 0.8fsc}
    \label{fig:outputrf08}
\end{figure}
\begin{figure}[htb]
    \centering
\begin{tikzpicture}
    \begin{axis}[
        legend pos = outer north east,
        legend style={font=\footnotesize},
        ylabel={power},
        xlabel={frequency},
        y unit={\deci\belm},
        x unit={\hertz},
        change x base,
        x SI prefix=giga,
        width=12cm,
        height=5cm,
        scale only axis,
        minor tick num = 1,
        ymin = -25
        ]
        \addplot[mark = +] table {data/rf/tx/1.0/0.0};
        \addplot[mark = x] table {data/rf/tx/1.0/10.0};
        \addplot[mark = asterisk] table {data/rf/tx/1.0/20.0};
        \addplot[mark = triangle] table {data/rf/tx/1.0/30.0};
        \addplot[mark = 10-pointed star] table {data/rf/tx/1.0/31.5};
        \legend{\SI{0}{dB},\SI{10}{dB},\SI{20}{dB},\SI{30}{dB},\SI{31.5}{dB}}
    \end{axis}
\end{tikzpicture}
    \caption{Output power, 1fsc}
    \label{fig:outputrf1}
\end{figure}
\begin{itemize}
    \item describe where stuff is comming from
\end{itemize}
\subsection{IF Frequency Response}
\subsubsection{TX}
\begin{itemize}
    \item 8Bit vs 16 Bit
\end{itemize}
\begin{figure}[htb]
    \centering
\begin{tikzpicture}
    \begin{axis}[
        ylabel={if frequency},
        xlabel={rf frequency},
        zlabel={output power},
        xlabel style={sloped},
        ylabel style={sloped},
        x unit={\hertz},
        change x base,
        x SI prefix=giga,
        y unit={\hertz},
        change y base,
        y SI prefix=mega,
        z unit={\deci\belm},
        width=12cm,
        height=5cm,
        scale only axis,
        minor tick num = 1,
        ]
        \addplot3[surf,mesh/cols=41] table {data/if/tx/mesh16_15.0};
    \end{axis}
\end{tikzpicture}
    \caption{RF/IF TX, 16 Bit, \SI{15.0}{\deci\bel} gain}
    \label{fig:rfiftx16}
\end{figure}

\begin{figure}[htb]
    \centering
\begin{tikzpicture}
    \begin{axis}[
        legend pos = outer north east,
        xlabel={if frequency},
        ylabel={output power},
        x unit={\hertz},
        change x base,
        x SI prefix=mega,
        y unit={\deci\belm},
        width=12cm,
        height=5cm,
        scale only axis,
        minor tick num = 1,
        ]
        \addplot[mark = +] table {data/if/tx/16_2000_15.0};
        \addplot[mark = star] table {data/if/tx/8_2000_15.0};
        \node (rechtsoben) at (axis description cs:1,1) {};
        \legend{16 Bit,8 Bit}
    \end{axis}
    \begin{axis}[
        at={(rechtsoben)},
        anchor=north east,
        scale only axis,
        minor tick num = 1,
        axis background/.style={fill=white},
        tick label style={font=\tiny},
        xmin = -3e6,
        xmax = 3e6,
        ymax = -8.6,
        ymin = -9.8,
        scaled x ticks={base 10:-6},
        xtick scale label code/.code={},
        width = 2cm
        ]
        \addplot[mark = +] table {data/if/tx/16_2000_15.0};
        \addplot[mark = star] table {data/if/tx/8_2000_15.0};
    \end{axis}
\end{tikzpicture}
    \caption{IF TX, \SI{15.0}{\deci\bel} gain, cfreq \SI{2}{\giga\hertz}}
    \label{fig:iftx}
\end{figure}

\subsection{Intermodulation Properties}
\subsubsection{RX}
\begin{itemize}
    \item SMBV - usrp did not work (seen IM caused by ADC in SMBV;)
    \item SMBV 10db dcblock - SMIQ 10db dcblock - pwr splitter - usrp
    \item probably caused by ADC
\end{itemize}
\begin{figure}[htb]
    \centering
\begin{tikzpicture}
    \begin{axis}[
        legend pos = outer north east,
        legend style={font=\footnotesize},
        ylabel={output power},
        xlabel={input power},
        y unit={\deci\belfs},
        x unit={\deci\belm},
        width=12cm,
        height=5cm,
        scale only axis,
        minor tick num = 1,
        ]
        \addplot[only marks, mark = +] table {data/ip3/rx/la};
        \addplot[only marks, mark = x] table {data/ip3/rx/im3};
        \addplot[only marks, mark = asterisk] table {data/ip3/rx/pim3};
        \addplot[domain = -93:-35, samples = 201,dotted] {1*x+33.0765};
        \addplot[domain = -93:-35, samples = 201,loosely dotted] {3*x+111.2541};
        \legend{$out$,$IM_3$,$IP_3$}
    \end{axis}
\end{tikzpicture}
    \caption{RX IP3}
    \label{fig:rxip3}
\end{figure}
\subsubsection{TX}
\begin{itemize}
    \item usrp - 16dB att - zvl
    \item gain stepping did not work; nor stepping zvl -> probably caused by ADC
\end{itemize}
\begin{figure}[htb]
    \centering
\begin{tikzpicture}
    \begin{axis}[
        legend pos = outer north east,
        legend style={font=\footnotesize},
        ylabel={output power},
        xlabel={input power},
        x unit={\deci\belfs},
        y unit={\deci\belm},
        width=12cm,
        height=5cm,
        scale only axis,
        minor tick num = 1,
        ]
        \addplot[only marks, mark = +] table {data/ip3/tx/la};
        \addplot[only marks, mark = x] table {data/ip3/tx/im3};
        \addplot[only marks, mark = asterisk] table {data/ip3/tx/pim3};
        \addplot[domain = -27:17, samples = 201,dotted] {1*x-4.2330};
        \addplot[domain = -27:17, samples = 201,loosely dotted] {3*x-33.6197};
        \legend{$out$,$IM_3$,$IP_3$}
    \end{axis}
\end{tikzpicture}
    \caption{TX IP3}
    \label{fig:txip3}
\end{figure}

\bibliographystyle{plain}
\bibliography{literature}

\end{document}
